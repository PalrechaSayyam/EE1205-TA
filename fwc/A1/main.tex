\documentclass[12pt,-letter paper]{article}

%\usepackage[left=1.5in,right=1in,top=1in,bottom=1in]{geometry}
%\usepackage[left=1.5in,right=1in]{geometry}
%\usepackage{geometry}
%\makeatletter%
%\textheight     243.5mm
%\textwidth      183.0mm
%\textwidth=31pc%
%\textheight=48pc
\usepackage{lipsum}% this package is included to get dummy paragraphs for sample purpose.
\usepackage{ulem}
\usepackage{alltt}
\usepackage{tfrupee}
\usepackage[anticlockwise,figuresright]{rotating}
\usepackage{pstricks}
\usepackage{wrapfig}
\usepackage{pstcol,pst-grad}
 \usepackage{bm}
\usepackage{enumitem}
\usepackage{listings}
    \usepackage{color}                                            %%
    \usepackage{array}                                            %%
    \usepackage{longtable}                                        %%
    \usepackage{calc}                                             %%
    \usepackage{multirow}                                         %%
    \usepackage{hhline}                                           %%
    \usepackage{ifthen}                                           %%
  %optionally (for landscape tables embedded in another document): %%
    \usepackage{lscape}     
    \usepackage{gensymb}     
    \usepackage{tabularx}
\usepackage{ifthen}%
\usepackage{amsmath}%
\usepackage{color}%
\usepackage{float}%
\usepackage{graphicx}%
%\usepackage[right]{showlabels}%
\usepackage{boites}%
\usepackage{boites_exemples}%
\usepackage{graphicx,pstricks}
%\usepackage{enumerate}%
\usepackage{latexsym}
\usepackage[fleqn]{mathtools}
\usepackage{amssymb,amsfonts,amsthm}
\usepackage{mathrsfs,makeidx,listings,verbatim,moreverb}
%%\usepackage{amsthm,mathrsfs,makeidx,listings,verbatim,moreverb}
%\let\eqref\ref%  updated on 20th April 2017

\usepackage{hyperref}%
%\usepackage[dvips]{hyperref}%
\hypersetup{bookmarksopen=false}%
\usepackage{breakurl}%
\usepackage{tkz-euclide} % loads  TikZ and tkz-base

\newcommand{\solution}{\noindent \textbf{Solution: }}
\providecommand{\mbf}{\mathbf}
\providecommand{\rank}{\text{rank}}
\providecommand{\pr}[1]{\ensuremath{\Pr\left(#1\right)}}
\providecommand{\qfunc}[1]{\ensuremath{Q\left(#1\right)}}
\providecommand{\sbrak}[1]{\ensuremath{{}\left[#1\right]}}
\providecommand{\lsbrak}[1]{\ensuremath{{}\left[#1\right.}}
\providecommand{\rsbrak}[1]{\ensuremath{{}\left.#1\right]}}
\providecommand{\brak}[1]{\ensuremath{\left(#1\right)}}
\providecommand{\lbrak}[1]{\ensuremath{\left(#1\right.}}
\providecommand{\rbrak}[1]{\ensuremath{\left.#1\right)}}
\providecommand{\cbrak}[1]{\ensuremath{\left\{#1\right\}}}
\providecommand{\lcbrak}[1]{\ensuremath{\left\{#1\right.}}
\providecommand{\rcbrak}[1]{\ensuremath{\left.#1\right\}}}
\newenvironment{amatrix}[1]{%
  \left(\begin{array}{@{}*{#1}{c}|c@{}}
}{%
  \end{array}\right)
}
\theoremstyle{remark}
\newtheorem{rem}{Remark}
\newtheorem{theorem}{Theorem}[section]
\newtheorem{problem}{Problem}
\newtheorem{proposition}{Proposition}[section]
\newtheorem{lemma}{Lemma}[section]
\newtheorem{corollary}[theorem]{Corollary}
\newtheorem{example}{Example}[section]
\newtheorem{definition}[problem]{Definition}
\newcommand{\sgn}{\mathop{\mathrm{sgn}}}
\providecommand{\abs}[1]{\left\vert#1\right\vert}
\providecommand{\res}[1]{\Res\displaylimits_{#1}} 
\providecommand{\norm}[1]{\left\lVert#1\right\rVert}
%\providecommand{\norm}[1]{\lVert#1\rVert}
\providecommand{\mtx}[1]{\mathbf{#1}}
\providecommand{\mean}[1]{E\left[ #1 \right]}
\providecommand{\fourier}{\overset{\mathcal{F}}{ \rightleftharpoons}}
%\providecommand{\hilbert}{\overset{\mathcal{H}}{ \rightleftharpoons}}
\providecommand{\system}{\overset{\mathcal{H}}{ \longleftrightarrow}}
 %\newcommand{\solution}[2]{\textbf{Solution:}{#1}}
%\newcommand{\solution}{\noindent \textbf{Solution: }}
\newcommand{\cosec}{\,\text{cosec}\,}
\providecommand{\dec}[2]{\ensuremath{\overset{#1}{\underset{#2}{\gtrless}}}}
\newcommand{\myvec}[1]{\ensuremath{\begin{pmatrix}#1\end{pmatrix}}}
\newcommand{\myaugvec}[2]{\ensuremath{\begin{amatrix}{#1}#2\end{amatrix}}}
\newcommand{\mydet}[1]{\ensuremath{\begin{vmatrix}#1\end{vmatrix}}}
\newcommand\figref{Fig.~\ref}
\newcommand\appref{Appendix~\ref}
\newcommand\tabref{Table~\ref}
\newcommand{\romanNumeral}[1]{\uppercase\expandafter{\romannumeral#1}}
%\numberwithin{equation}{section}
%\numberwithin{equation}{subsection}
%\numberwithin{problem}{section}
%\numberwithin{definition}{section}
%\makeatletter
%\@addtoreset{figure}{problem}
%\makeatother

%\let\StandardTheFigure\thefigure
\let\vec\mathbf
\def\inputGnumericTable{}                                 %%
%New macro definitions
\newcounter{matchleft}\newcounter{matchright}

\newenvironment{matchtabular}{%
  \setcounter{matchleft}{0}%
  \setcounter{matchright}{0}%
  \tabularx{\textwidth}{%
    >{\leavevmode\hbox to 1.5em{\stepcounter{matchleft}\arabic{matchleft}.}}X%
    >{\leavevmode\hbox to 1.5em{\stepcounter{matchright}\alph{matchright})}}X%
    }%
}{\endtabularx}


%\bibliographystyle{ieeetr}
\setlength{\parindent}{0pt}
\begin{document}
\bibliographystyle{IEEEtran}

\vspace{3cm}

%\title{
%	\logo{
%FWC Assignment-1
%	}
%}
%\author{ Sayyam Palrecha$^{*}$ FWC22248% <-this % stops a space
%	\thanks{}
	
%}
\title{CBSE Class 12, 2016, 065/SET2/N}
\date{\today}

% make the title area
\maketitle

%\tableofcontents

\bigskip

\renewcommand{\thefigure}{\theenumi}
\renewcommand{\thetable}{\theenumi}
%\renewcommand{\theequation}{\theenumi}

\begin{enumerate}

\begin{center} $\vec{SECTION-A}$ \\ \end{center}
\item If $\vec{a} = 4\vec{i} -\vec{j} + \vec{k}$ and $\vec{b} = 2\vec{i} -2\vec{j} + \vec{k} $, then find a unit vector parallel to the vector $\vec{a} + \vec{b} $.\\

\item Find $\lambda$ and $\mu$ if\\
$\brak{\vec{i} + 3\vec{j} + 9\vec{k}}\times\brak{3\vec{i} - \lambda\vec{j} + \mu\vec{k}} = \vec{0}$\\

\item Write the sum of intercepts cut by the plane $\vec{r}\cdot\brak{2\vec{i} + \vec{j} - \vec{k}} - 5 = 0$ on the three axes.\\

\item For what values of $k$, the system of linear equations
\begin{align*}
x+y+z &= 2\\
2x+y+z &= 3\\
3x+2y+kz &= 4
\end{align*}
has a unique solution?\\

\item If $A$ is $3\times 3$ matrix and $\abs{3A} = k\abs{A}$, then write the value of $k$.\\

\item If $A = \myvec{\cos{\alpha} & \sin{\alpha}\\ -\sin{\alpha} & \cos{\alpha}}$, find $alpha$ satisfying $0 < \alpha < \frac{\pi}{2}$ when $A + A^T = \sqrt{2}I_2$; where $A^T$ is transpose of $A$.\\ 

\begin{center} $\vec{SECTION-B}$ \\ \end{center}

\item A bag $X$ contains 4 white balls and 2 black balls, while another bag $Y$ contains 3 white balls and 3 black balls. Two balls are drawn (without replacement) at random from one of the bags and were found to be one white and one black. Find the probability that the balls were drawn from bag $Y$.
\begin{center} $\vec{OR}$ \\ \end{center}
A and B throw a pair of dice alternately, till one of them gets a total of 10 and wins the game. Find their respective probabilities of winning, if A starts first.\\

\item Find the coordinates of the foot of perpendicular drawn from the point $A\brak{-1,8,4}$ to the line joining the points $B\brak{0,-1,3}$ and $A\brak{2,3,-1}$. Hence find the image of the point A in the line $BC$.\\

\item Show that the four point $A\brak{4, 5, 1}, B\brak{0,-1,-1}, C\brak{3,9,4} \text{ and } D\brak{-4,4,4}$ are coplanar.\\

\item Find the particular solution of the differential equation
\begin{align*}
2ye^{\frac{x}{y}}dx + \brak{y - 2xe^{\frac{x}{y}}}dy = 0
\end{align*}
given that $x=0$ when $y=1$.\\
\item Find the particular solution of differential equation: $\frac{dy}{dx} = -\frac{x+y\cos{x}}{1+\sin{x}}$ given that $y = 1$ when $x=0$.\\

\item Find: $\int{\brak{x+3}\sqrt{3 - 4x - x^2}dx}$.\\

\item Find: $\int{\frac{(2x - 5)e^{2x}}{(2x-3)^3}dx}$
\begin{center} $\vec{OR}$ \\ \end{center}
Find: $\int{\frac{x^2+x+1}{(x^2+1)(x+2)}dx}$\\

\item Find the equation of tangents to the curve $y=x^3+2x-4$, which are perpenicular to line $x+14y+3=0$.\\

\item If $x\cos(a+y) = \cos{y}$ then prove that $\frac{dy}{dx} = \frac{\cos^2(a+y)}{\sin{a}}$. Hence show that $\sin^2(a+y)\frac{dy}{dx} = 0$.
\begin{center} $\vec{OR}$ \\ \end{center}
Find $\frac{dy}{dx}$ if $y = \sin^{-1}\brak{\frac{6x - 4\sqrt{1-4x^2}}{5}}$\\

\item Evaluate: $\int_{-2}^{2}\frac{x^2}{1+5^x}dx$.\\

\item If \begin{align}f(x) &= \begin{cases}\frac{\sin(a+1)x + 2\sin x}{x}, &x<0\\ 2, &x=0 \\ \frac{\sqrt{1+bx}-1}{x}, &x>0 \end{cases}\end{align} is continuous at $x=0$, then find the values of $a$ and $b$.\\

\item A typist charges \rupee 145 for typing 10 English and 3 Hindi pages, while charges for typing 3 English and 10 Hindi pages are \rupee 180. Using matrices, find the charges of typing one English and one English page separately. However typist charged only \rupee 2 per page from a poor student Shyam for 5 Hindi pages. How much less was charged from this peeo boy? Which values are reflected in this problem?\\

\item Solve for x: $\tan^{-1}(x-1) + \tan^{-1}(x+1) = \tan^{-1}(3x)$.
\begin{center} $\vec{OR}$ \\ \end{center}
Prove that $\tan^{-1}\brak{\frac{6x-8x^3}{1-12x^2}} + \tan^{-1}\brak{\frac{4x}{1-4x^2}} = \tan^{-1}2x; \abs{2x}<\frac{1}{\sqrt{3}}$.\\

\item Using the method of integration, find the area of the triangular region whose vertices are (2,-2), (4,-3) and (1,2).\\

\begin{center} $\vec{SECTION-C}$ \\ \end{center}

\item Using properties of determinants, prove that
\begin{align*}\mydet{\brak{x+y}^2 & zx & zy \\ zx & \brak{z+y}^2 & xy \\ zy & xy & \brak{z+x}^2} = 2xyz\brak{x+y+z}^3
\end{align*}
\begin{center} $\vec{OR}$ \\ \end{center}
If $A = \myvec{1 & 0 & 2 \\ 0 & 2 & 1 \\ 2 & 0 & 3}$ and $A^3 - 6A^2 + 7A + kI_3 = O$ find $k$.

\item A retired person wants to invest an amount of \rupee 50,000. His broker recommends investing in the type of bonds 'A' and 'B' yielding 10\% and 9\% return respectively on the invested amount. He decides to invest at least \rupee 20,000 in bond 'A' and at least \rupee 10,000 in bond 'B'. He also wants to invest at least as much in bond 'A' as in bond 'B'. Solve this linear programming problem graphically to maximise his returns.\\

\item Find the equations of the plane which contains the line of intersection of the planes
\begin{align*}\vec{r}\cdot\brak{\vec{i} -2\vec{j} + 3\vec{k}}-4&=0 \\ \vec{r}\cdot\brak{-2\vec{i} + \vec{j} + \vec{k}}+5&=0 
\end{align*}
and whose intercept on x-axis is equal to that of y-axis.\\

\item Prove that $y = \frac{4\sin\theta}{2+\cos\theta} - \theta$ is an increasing function of $\theta$ on $\left [0, \frac{\pi}{2}\right ]$.
\begin{center} $\vec{OR}$ \\ \end{center}
Show that semi-vertical angle of a cone of a maximum volume and given slant height is $\cos^{-1}\brak{\frac{1}{\sqrt{3}}}$.

\item Let $A=R\times R$ and * be a binary operation on $A$ defined by $(a,b)*(c,d) = (a+c,b+d)$\\
Show that * is commutative and associative. Find the identity elemeny for * on $A$. Also find the inverse of every element $(a, b) \in A$.\\

\item Three numbers are selected at random (without replacement) from first six positive integers. Let $X$ denote the largest of the three numbers obtained. Find the probability distribution of $X$. Also, find the mean and variance of the distribution.

\end{enumerate}

\end{document}
